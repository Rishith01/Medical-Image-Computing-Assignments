\documentclass[11pt]{article}

\usepackage{amsmath,amssymb}
\usepackage{graphicx}
\usepackage{subcaption}
\usepackage{float}
\usepackage{booktabs}
\usepackage{geometry}
\geometry{margin=1in}
\hyphenpenalty = 10000

\title{Bayesian Image Denoising Using Markov Random Field Priors}
\author{
Rishith Gupta (23B1234) \\
Shivam Chaubey (23B1244)
}

\date{}

\begin{document}
\maketitle

\section*{1. Bayesian Denoising of a Phantom MR Image}

\subsection*{Noise Model}
The observed noisy image is modeled using an independent and identically distributed (i.i.d.) Gaussian noise model. This choice is motivated by the fact that Gaussian noise provides a good approximation to thermal noise in MR magnitude images and leads to a convex likelihood term, enabling stable gradient-based optimization.

\subsection*{MRF Prior Models}
A 4-neighborhood Markov Random Field (MRF) prior with pairwise cliques was employed. The following potential functions were used:

\begin{itemize}
\item Quadratic prior:
\[
g_1(u) = u^2
\]

\item Huber prior:
\[
g_2(u) =
\begin{cases}
\frac{1}{2}u^2, & |u| \le \gamma \\
\gamma|u| - \frac{1}{2}\gamma^2, & |u| > \gamma
\end{cases}
\]

\item Discontinuity-adaptive prior:
\[
g_3(u) = \gamma|u| - \gamma^2 \log\left(1 + \frac{|u|}{\gamma}\right)
\]
\end{itemize}

The quadratic prior enforces global smoothness, while the Huber and discontinuity-adaptive priors reduce penalization of large intensity differences, enabling edge preservation.

\subsection*{Optimization Method}
Maximum-a-posteriori (MAP) estimation was performed using gradient ascent with a dynamically adapted step size. The noisy image was used as the initial estimate. The step size was adjusted to ensure monotonic increase of the log-posterior objective function at every iteration, thereby guaranteeing convergence.

\subsection*{RRMSE Metric}
The reconstruction quality was evaluated using the relative root-mean-squared error (RRMSE), defined as:
\[
\text{RRMSE}(A,B) =
\sqrt{
\frac{\sum_p (A(p) - B(p))^2}{\sum_p A(p)^2}
}
\]
where \(A\) denotes the noiseless image and \(B\) denotes the estimated image.

\subsection*{RRMSE of Noisy Image}
The RRMSE between the noisy and noiseless images was computed prior to denoising and serves as a baseline for performance comparison.

\subsection*{Optimal Parameter Selection}
For each prior, parameters were tuned to minimize the RRMSE. The optimal parameters were obtained via systematic search and validated using local perturbation analysis.

\begin{table}[H]
\centering
\begin{tabular}{lccc}
\toprule
\textbf{Prior} & $\alpha^*$ & $\gamma^*$ & \textbf{RRMSE} \\
\midrule
Quadratic & 0.059752 & -- & 0.280725 \\
Huber & 0.988762 & 0.031240 & 0.239420 \\
Discontinuity-adaptive & 0.999784 & 0.038758 & 0.243603 \\
\bottomrule
\end{tabular}
\caption{Optimal parameters and corresponding RRMSE values}
\end{table}



\begin{table}[H]
\centering
\begin{tabular}{lccccc}
\toprule
\textbf{Prior} &
RRMSE($\alpha^*,\gamma^*$) &
RRMSE($1.2\alpha^*$) &
RRMSE($0.8\alpha^*$) &
RRMSE($1.2\gamma^*$) &
RRMSE($0.8\gamma^*$) \\
\midrule
Quadratic
& 0.280725
& 0.281167
& 0.281199
& --
& -- \\

Huber
& 0.239420
& 0.239421
& 0.241363
& 0.239898
& 0.240137 \\

Discontinuity-adaptive
& 0.243603
& 0.243088
& 0.246478
& 0.243956
& 0.244195 \\
\bottomrule
\end{tabular}
\caption{RRMSE values for $\pm20\%$ perturbations of optimal parameters}
\end{table}


\textbf{Justification:}  
In all cases, perturbing the optimal parameters resulted in higher RRMSE values. This confirms that the reported parameter values correspond to local minima of the reconstruction error and therefore represent optimal tuning.

\subsection*{Denoised Image Results}

\begin{figure}[H]
\centering
\includegraphics[width=\textwidth]{Noiseless_vs_Noisy_Denoised_Q1.png}
\caption{(i) Noiseless image, (ii) Noisy image, (iii) Denoised image using quadratic prior, (iv) Denoised image using Huber prior, (v) Denoised image using discontinuity-adaptive prior. All images are displayed using the same colormap.}
\end{figure}

\subsection*{Objective Function Convergence}

\begin{figure}[H]
\centering
\includegraphics[width=0.8\textwidth]{quadratic_prior_Convergence_curve_Q1.png}
\caption{Objective function value versus iteration for quadratic prior using optimal parameters.}
\end{figure}

\begin{figure}[H]
\centering
\includegraphics[width=0.8\textwidth]{huber_prior_Convergence_curve_Q1.png}
\caption{Objective function value versus iteration for Huber prior using optimal parameters.}
\end{figure}

\begin{figure}[H]
\centering
\includegraphics[width=0.8\textwidth]{Discontinuity_prior_Convergence_curve_Q1.png}
\caption{Objective function value versus iteration for discontinuity-adaptive prior using optimal parameters.}
\end{figure}

\textbf{Justification:}  
The objective function increases monotonically with iteration for all three priors, confirming correct implementation of gradient ascent with adaptive step size and convergence to a MAP estimate.

\section*{2. Bayesian Denoising of Brain MRI}
The same denoising framework was applied to the brain MRI image. Similar trends were observed: the quadratic prior produced excessive smoothing, while the Huber and discontinuity-adaptive priors preserved anatomical edges more effectively.

\begin{table}[H]
\centering
\begin{tabular}{lccc}
\toprule
\textbf{Prior} & $\alpha^*$ & $\gamma^*$ & \textbf{RRMSE} \\
\midrule
Quadratic & 0.079704 & -- & 0.121299 \\
Huber & 0.695953 & 0.030194 & 0.113147 \\
Discontinuity-adaptive & 0.998209 & 0.023875 & 0.113360 \\
\bottomrule
\end{tabular}
\caption{Optimal parameters and corresponding RRMSE values}
\end{table}


\begin{table}[H]
\centering
\begin{tabular}{lccccc}
\toprule
\textbf{Prior} &
RRMSE($\alpha^*,\gamma^*$) &
RRMSE($1.2\alpha^*$) &
RRMSE($0.8\alpha^*$) &
RRMSE($1.2\gamma^*$) &
RRMSE($0.8\gamma^*$) \\
\midrule
Quadratic
& 0.121299
& 0.121765
& 0.121873
& --
& -- \\

Huber
& 0.113147
& 0.113924
& 0.114345
& 0.113601
& 0.114006 \\

Discontinuity-adaptive
& 0.113360
& 0.114165
& 0.114869
& 0.113671
& 0.114048 \\
\bottomrule
\end{tabular}
\caption{RRMSE values for $\pm20\%$ perturbations of optimal parameters}
\end{table}


\begin{figure}[H]
\centering
\includegraphics[width=\textwidth]{Noiseless_vs_Noisy_Denoised_Q2.png}
\caption{(i) Noiseless image, (ii) Noisy image, (iii) Denoised image using quadratic prior, (iv) Denoised image using Huber prior, (v) Denoised image using discontinuity-adaptive prior. All images are displayed using the same colormap.}
\end{figure}

\subsection*{Objective Function Convergence}

\begin{figure}[H]
\centering
\includegraphics[width=0.8\textwidth]{quadratic_prior_Convergence_curve_Q2.png}
\caption{Objective function value versus iteration for quadratic prior using optimal parameters.}
\end{figure}

\begin{figure}[H]
\centering
\includegraphics[width=0.8\textwidth]{huber_prior_Convergence_curve_Q2.png}
\caption{Objective function value versus iteration for Huber prior using optimal parameters.}
\end{figure}

\begin{figure}[H]
\centering
\includegraphics[width=0.8\textwidth]{Discontinuity_prior_Convergence_curve_Q2.png}
\caption{Objective function value versus iteration for discontinuity-adaptive prior using optimal parameters.}
\end{figure}

\section*{3. Bayesian Denoising of RGB Microscopy Image}
The same Bayesian denoising framework was applied to a vector-valued RGB microscopy image. The denoising was performed using three vector-valued priors corresponding to different regularization strategies. Performance was evaluated quantitatively using RRMSE and qualitatively using visual inspection. Unlike the grayscale images in Part 2, this experiment required vector-valued denoising, where regularization was applied jointly across the RGB channels using vector-valued MRF priors.

\begin{table}[H]
\centering
\begin{tabular}{lccc}
\toprule
\textbf{Prior} & $\alpha^*$ & $\gamma^*$ & \textbf{RRMSE} \\
\midrule
Quadratic & 0.038774 & -- & 0.122346 \\
Huber & 0.465101 & 0.012349 & 0.113304 \\
Discontinuity-adaptive & 0.514589 & 0.010655 & 0.113690 \\
\bottomrule
\end{tabular}
\caption{Optimal parameters and corresponding RRMSE values}
\end{table}

\begin{table}[H]
\centering
\begin{tabular}{lccccc}
\toprule
\textbf{Prior} &
RRMSE($\alpha^*,\gamma^*$) &
RRMSE($1.2\alpha^*$) &
RRMSE($0.8\alpha^*$) &
RRMSE($1.2\gamma^*$) &
RRMSE($0.8\gamma^*$) \\
\midrule
Quadratic
& 0.122346
& 0.122888
& 0.122602
& --
& -- \\

Huber
& 0.113304
& 0.116376
& 0.115039
& 0.114111
& 0.113808 \\

Discontinuity-adaptive
& 0.113690
& 0.115537
& 0.116700
& 0.113792
& 0.114593 \\
\bottomrule
\end{tabular}
\caption{RRMSE values for $\pm20\%$ perturbations of optimal parameters}
\end{table}

\begin{figure}[H]
\centering
\includegraphics[width=\textwidth]{Noiseless_vs_Noisy_Denoised_Q2.png}
\caption{(i) Noiseless image, (ii) Noisy image, (iii) Denoised image using quadratic prior, (iv) Denoised image using Huber prior, (v) Denoised image using discontinuity-adaptive prior. All images are displayed using the same colormap.}
\end{figure}

\subsection*{Objective Function Convergence}

\begin{figure}[H]
\centering
\includegraphics[width=0.8\textwidth]{quadratic_prior_Convergence_curve_Q2.png}
\caption{Objective function value versus iteration for quadratic prior using optimal parameters.}
\end{figure}

\begin{figure}[H]
\centering
\includegraphics[width=0.8\textwidth]{huber_prior_Convergence_curve_Q2.png}
\caption{Objective function value versus iteration for Huber prior using optimal parameters.}
\end{figure}

\begin{figure}[H]
\centering
\includegraphics[width=0.8\textwidth]{Discontinuity_prior_Convergence_curve_Q2.png}
\caption{Objective function value versus iteration for discontinuity-adaptive prior using optimal parameters.}
\end{figure}

\subsection*{Edge Preservation Analysis}
Among the three vector-valued priors:
\begin{itemize}
\item The squared $\ell_2$ prior excessively smooths edges.
\item The $\ell_2$ norm prior preserves edges better by reducing penalization of large gradients.
\item The Huber-regularized $\ell_1$ prior provides the best edge preservation by combining robustness to outliers with smoothness in homogeneous regions.
\end{itemize}

\textbf{Justification:}  
Edge preservation arises from non-quadratic penalties that limit smoothing across high-contrast boundaries while still suppressing noise in flat regions.

% \begin{figure}[H]
% \centering
% \includegraphics[width=\textwidth]{rgb_results_placeholder.png}
% \caption{RGB microscopy denoising results using three different priors.}
% \end{figure}

\section*{4. Dictionary Learning on Image Patches and Image Denoising}

\subsection*{Dictionary Learning Formulation}
Let $\{x_i\}_{i=1}^I$ denote a set of image patches of size $8 \times 8$, vectorized into $\mathbb{R}^{64}$.  
We consider the following dictionary learning problem:
\[
\min_{D,\{r_i\}} \sum_{i=1}^I \|x_i - D r_i\|_2^2 + \lambda \|r_i\|_p^p
\quad \text{subject to } \|d_k\|_2 \le 1 \;\; \forall k,
\]
where $D \in \mathbb{R}^{64 \times 64}$ is the dictionary whose columns $d_k$ are atoms, and $r_i$ are sparse coefficient vectors.

The hard constraint on atom norms prevents trivial scaling ambiguities between $D$ and $r_i$ and stabilizes optimization.

\subsection*{Patch Selection Strategy}
For computational efficiency and to encourage learning informative structures, only patches with sufficiently large intensity variance were used for dictionary learning. Constant or near-constant patches were excluded.

\textbf{Justification:}  
High-variance patches are more likely to contain edges, textures, and anatomical structures. Training on such patches improves the expressiveness of the learned dictionary while avoiding redundant atoms corresponding to flat regions.

\subsection*{Optimization Strategy}
The optimization was performed using alternating minimization:
\begin{itemize}
    \item Fix $D$ and update $\{r_i\}$ using iterative shrinkage depending on the value of $p$.
    \item Fix $\{r_i\}$ and update $D$ using projected gradient descent, followed by normalization to satisfy $\|d_k\|_2 \le 1$.
\end{itemize}

This strategy was used consistently for all values of $p$.

\subsection*{Effect of the Parameter $p$}
Dictionaries were learned for $p = 2, 1.6, 1.2$ and $0.8$.

\textbf{Justification:}  
As $p$ decreases below $2$, the penalty increasingly promotes sparsity in the coefficients. Lower values of $p$ encourage representations using fewer active atoms, leading to more localized and edge-like dictionary elements.

\subsection*{Objective Function Convergence}

\begin{figure}[H]
\centering
\includegraphics[width=0.8\textwidth]{objective_all_p.png}
\caption{Objective function value versus iteration for dictionary learning with $p = 2, 1.6, 1.2,$ and $0.8$.}
\end{figure}

\textbf{Observation and Justification:}  
For all values of $p$, the objective function decreases monotonically and converges. Lower values of $p$ result in slower convergence due to increased non-convexity, but yield sparser representations.

\subsection*{Dictionary Atoms: Before and After Optimization}

\begin{figure}[H]
\centering

% ---------- Initial atoms ----------
\begin{subfigure}{0.24\textwidth}
    \includegraphics[width=\linewidth]{initial_atom_p0.8.png}
    \caption{$p=0.8$}
\end{subfigure}
\begin{subfigure}{0.24\textwidth}
    \includegraphics[width=\linewidth]{initial_atom_p1.2.png}
    \caption{$p=1.2$}
\end{subfigure}
\begin{subfigure}{0.24\textwidth}
    \includegraphics[width=\linewidth]{initial_atom_p1.6.png}
    \caption{$p=1.6$}
\end{subfigure}
\begin{subfigure}{0.24\textwidth}
    \includegraphics[width=\linewidth]{initial_atom_p2.png}
    \caption{$p=2.0$}
\end{subfigure}

\vspace{0.5em}

% ---------- Learned atoms ----------
\begin{subfigure}{0.24\textwidth}
    \includegraphics[width=\linewidth]{learned_atom_p0.8.png}
    \caption{$p=0.8$}
\end{subfigure}
\begin{subfigure}{0.24\textwidth}
    \includegraphics[width=\linewidth]{learned_atom_p1.2.png}
    \caption{$p=1.2$}
\end{subfigure}
\begin{subfigure}{0.24\textwidth}
    \includegraphics[width=\linewidth]{learned_atom_p1.6.png}
    \caption{$p=1.6$}
\end{subfigure}
\begin{subfigure}{0.24\textwidth}
    \includegraphics[width=\linewidth]{learned_atom_p2.png}
    \caption{$p=2.0$}
\end{subfigure}

\caption{Initial (top row) and learned (bottom row) dictionary atoms for different values of $p$.}
\end{figure}


\textbf{Observation and Justification:}  
For $p=2$, atoms resemble smooth, global patterns. As $p$ decreases, atoms become more localized and edge-like. This behavior is expected since sparser models favor atoms that capture distinct directional structures.

\subsection*{Coefficient Histogram Analysis}

\begin{figure}[H]
\centering
\includegraphics[width=0.8\textwidth]{histogram_coeffs.png}
\caption{Histogram of coefficients pooled across all patches for different values of $p$.}
\end{figure}

\textbf{Observation and Justification:}  
As $p$ decreases, the coefficient distribution becomes more peaked at zero with heavier tails, indicating increased sparsity. This aligns with theoretical expectations for $\ell_p$ penalties with $p<2$.

\subsection*{Image Denoising Using the Learned Dictionary ($p=0.8$)}
A noisy version of the chest CT image was generated by adding i.i.d. Gaussian noise with standard deviation equal to $10\%$ 
of the image intensity range.

\begin{figure}[H]
\centering
\includegraphics[width=0.8\textwidth]{noisy_image.png}
\caption{Noisy Image with 10\% Gaussian noise
}
\end{figure}


The denoising problem was formulated as:
\[
\min_x \|y - x\|_2^2 + \mu \sum_i \|r_i(x)\|_p^p,
\]
where $r_i(x)$ denotes the sparse representation of overlapping patches of $x$ using the learned dictionary.

\subsection*{Denoising Convergence}

\begin{figure}[H]
\centering
\includegraphics[width=0.8\textwidth]{denoise_objective.png}
\caption{Objective function value versus iteration during dictionary-based image denoising.}
\end{figure}

\textbf{Justification:}  
The monotonic decrease of the objective function confirms stable convergence of the optimization process.

\subsection*{Denoising Results}

\begin{figure}[H]
\centering
\includegraphics[width=0.9\textwidth]{ct_denoising.png}
\caption{Left: original chest CT image, center: simulated noisy image, right: denoised image using the learned dictionary with $p=0.8$.}
\end{figure}

\textbf{Conclusion:}  
Dictionary-based denoising using a sparsity-promoting prior effectively suppresses noise while preserving fine anatomical details. The learned dictionary captures meaningful image structures and generalizes well beyond the training patches.

\end{document}
